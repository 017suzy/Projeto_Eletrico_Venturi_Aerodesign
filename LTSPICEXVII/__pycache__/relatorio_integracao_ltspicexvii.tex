
\documentclass[a4paper,12pt]{article}
\usepackage[utf8]{inputenc}
\usepackage[T1]{fontenc}
\usepackage[brazil]{babel}
\usepackage{hyperref}
\usepackage{geometry}
\geometry{margin=2.5cm}

\title{Integração LTSpice e Python}
\author{Relatório Técnico}
\date{\today}

\begin{document}

\maketitle

\section{Integração LTSpice e Python}

\subsection*{Observações Importantes}

Foram utilizadas as seguintes fontes como referência para a realização desta tarefa:

\begin{itemize}
    \item \href{https://www.youtube.com/watch?v=BIkMCZUO-Kc}{\texttt{https://www.youtube.com/watch?v=BIkMCZUO-Kc}} \\
    (utilizado para seguir corretamente cada passo e verificar os resultados esperados);

    \item \href{https://acidbourbon.wordpress.com/2019/11/26/seamless-integration-of-ltspice-in-python-numpy-signal-processing/}{\texttt{https://acidbourbon.wordpress.com/2019/11/26/seamless-integration-of-ltspice-in-python-numpy-signal-processing/}} \\
    (artigo contendo todos os arquivos-fonte utilizados na integração).
\end{itemize}

É necessário colocar todos os arquivos baixados do artigo em um único 
diretório (ou seja, na mesma pasta), para que os códigos executem corretamente.

Além disso, é muito importante \textbf{não modificar o arquivo Python \texttt{apply\_ltspice\_filter.py}}, pois ele é o 
responsável por realizar a ligação entre o \textbf{LTSpice XVII} e a linguagem Python. 

\paragraph{}
Observação: nesta integração utilizou-se o \textbf{LTSpice XVII}, e não a versão comum do LTSpice.

\paragraph{}
Outro ponto essencial é o arquivo \texttt{.asc}, que deve ter o mesmo nome registrado tanto 
no \texttt{apply\_ltspice\_filter.py} quanto no \texttt{filter\_demo.py}, para que todo o conjunto 
de códigos funcione corretamente.

\subsection*{Descrição da Tarefa}

A tarefa consiste na leitura de dados diretamente do código Python (por exemplo, características do cabo) para o cálculo 
da \textbf{resistência total do cabo}. 

Em seguida, o valor calculado é enviado ao \textbf{LTSpice XVII} como o parâmetro \texttt{R\_cabo}, que é atribuído ao resistor 
de mesmo nome dentro do circuito. Dessa forma, o LTSpice configura automaticamente a resistência 
desse componente com base no valor calculado e mede a tensão final após essa resistência, 
correspondendo à tensão que chega ao servo.

\paragraph{}
O arquivo \texttt{filter\_circuit.asc} foi modificado ao final para se ajustar ao 
código desenvolvido para esta tarefa, garantindo compatibilidade completa entre a simulação no 
LTSpice e o processamento em Python.

\end{document}
